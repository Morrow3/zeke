\part{Introduction}

Zeke is a portable operating system for ARM processors.\footnote{Zeke is
portable to at least in a sense of portability between different ARM cores
and architectures.} Figure \ref{figure:zeke} illustrates architectural layers
of Zeke. Scope of Zeke project is to implement a portable
Unix-like\footnote{Project is aiming to implement the core set of kernel
features specified in \ac{POSIX} standard.} operating system from
scratch\footnote{At least almost from scratch, some functionality is derived
from freeBSD as well as some libraries are taken from other sources.} that
is optimized for ARM architectures and is free of legacy.

In addition to portability another goal has been configurability and adjustable
footprint of the kernel binary. This is achieved by modular kernel architecture
and static compile time configuration. Like for example \ac{HAL} is already
almost fully locked in compilation phase.

\begin{figure}
  \pgfdeclarelayer{background}
\pgfsetlayers{background,main}

\definecolor{mybluei}{RGB}{124,156,205}
\definecolor{myblueii}{RGB}{73,121,193}
\definecolor{mygreen}{RGB}{202,217,126}
\definecolor{mypink}{RGB}{233,198,235}

% this length is used to control the width of the light blue frame
% for the upper part of the diagram
\newlength\myframesep
\setlength\myframesep{8pt}

\pgfkeys{
    /tikz/node distance/.append code={
        \pgfkeyssetvalue{/tikz/node distance value}{#1}
    }
}

\newcommand\widernode[5][blueb]{
\node[
        #1,
        inner sep=0pt,
        shift=($(#2.south)-(#2.north)$),
        yshift=-\pgfkeysvalueof{/tikz/node distance value},
        fit={(#2) (#3)},
        label=center:{\sffamily\bfseries\color{white}#4}] (#5) {};
}

\begin{tikzpicture}[node distance=3pt,outer sep=0pt,
blueb/.style={
  draw=white,
  fill=mybluei,
  rounded corners,
  text width=2.5cm,
  font={\sffamily\bfseries\color{white}},
  align=center,
  text height=12pt,
  text depth=9pt},
greenb/.style={blueb,fill=mygreen},
]
\node[blueb] (init) {init};
\node[blueb,right=of init] (usr) {User Program};
\widernode{init}{usr}{libc}{libc};
\widernode{libc}{libc}{Syscall Interface}{syscall};
\widernode{syscall}{syscall}{Kernel Core}{kernel};
\node[blueb, right=of kernel](vfs) {VFS};
\node[greenb,below=of kernel] (hal) {HAL};
\node[greenb,right=of hal] (drv) {Drivers};
\begin{pgfonlayer}{background}
\draw[blueb,draw=black,fill=mybluei!40] 
  ([xshift=-\myframesep,yshift=3\myframesep]current bounding box.north west) 
    rectangle 
  ([xshift=\myframesep,yshift=-\myframesep]current bounding box.south east);
\end{pgfonlayer}
\path  let \p1=( $ (syscall.east) - (syscall.west) $ )
 in node[blueb,inner xsep=0pt,draw=black,fill=myblueii,below=4pt of current bounding box.south,text width=\x1+2*\myframesep+2\pgflinewidth] (HW) {Hardware};
\node[font=\sffamily\itshape\color{white},above=of init] {Zeke};
\end{tikzpicture}
  \centering
  \caption{Zeke: a Portable Operating System.}
  \label{figure:zeke}
\end{figure}

\section*{List of Abbreviations and Meanings}

\begin{acronym}[UniversalSpacer]
\acro{POSIX}[POSIX]{Portable Operating System Interface for uniX\acroextra{ is a
  family of standards specifying a standard set of features for compatibility
  between operating systems.}}

\acro{vfs}[VFS]{Virtual File System\acroextra{.}}
\acro{inode}[inode]{\acroextra{inode is the file index node in Unix-style file
  systems that is used to represent a file file system object.}}
\acro{vnode}[vnode]{\acroextra{vnode is like inode but it's used as an
  abstraction level for compatibility between different file systems in Zeke.
  Particularly it's used at VFS level.}}
\acro{ramfs}[ramfs]{RAM file system\acroextra{.}}

\acro{dynmem}[dynmem]{(dynamic memory)\acroextra{ is the block memory allocator
  system in Zeke, allocating in 1 MB blocks.}}
\acro{vralloc}[VRAlloc]{Virtual Region Allocator\acroextra{.}}

\acro{PUnit}[PUnit]{a portable unit testing framework for C\acroextra{.}}
\acro{GCC}[GCC]{GNU Compiler Collection\acroextra{.}}
\acro{GLIBC}[GLIBC]{The GNU C Library\acroextra{.}}
\acro{CPU}[CPU]{Central Processing Unit\acroextra{ is, generally speaking, the
  main processor of some computer or embedded system.}}

\acro{MMU}[MMU]{Memory Management Unit\acroextra{.}}
\acro{DMA}[DMA]{Direct Memory Access\acroextra{ is a feature that allows
 hardware subsystems to commit memory transactions without \acsu{CPU}'s
 intervention.}}
\end{acronym}