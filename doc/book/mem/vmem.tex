\section{Introduction to vmem in Zeke}

Every process has its own master page table and varying number of L2 page
tables. Kernel has its own master page table too. Static/fixed entries are
copied to all master page tables created. Process shares its master page
table with its childs.

Process page tables are stored in dynmem area.

ARM note: Only 4 kB pages are used with L2 page tables so XN (Execute-Never) bit
is always usable also for L2 pages.

\subsection{Domains}

See \verb+MMU_DOM_xxx+ definitions.

\subsection{Virtual memory abstraction levels}

\begin{verbatim}
    U                      +---------------+
    S                      |    malloc     |
    R                      +---------------+
    -------------------------------|-------------
    K   +---------------+  +---------------+
    E   |    kmalloc    |  |      ???      |
    R   +---------------+  +---------------+
    N           |                  |
    E   +---------------+          |
    L   |    dynmem     |----------+
        +---------------+          |
                |                  |
        +---------------+          |
        |    mmu HAL    |----------+
        +---------------+
                |
        +-----------------------+
        | CPU specific MMU code |
        +-----------------------+
    ------------|--------------------------------
        +-------------------+
        | MMU & coProcessor |
        +-------------------+
\end{verbatim}

\begin{itemize}
  \item \verb+kmalloc+ - is a kernel level memory allocation service.
  \item \verb+dynmem+ - is a dynamic memory allocation system that allocates \&
        frees contiguous blocks of physical memory.
  \item  mmu HAL - is the abract MMU interface provided by \verb+mmu.h+
         and \verb+mmu.c+.
  \item CPU specific MMU code is the module responsible of configuring the
        physical MMU layer and implementing the interface prodived by
        \verb+mmu.c+
\end{itemize}


\section{dynmem}

Dynmem allocates 1MB sections from L1 kernel master page table and always
returns a physically contiguous memory region. If dynmem is passed for a thread
it can be mapped either as a section entry or via L2 page table.

