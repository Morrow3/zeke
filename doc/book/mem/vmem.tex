\chapter{Virtual Memory}

\section{Introduction to vmem in Zeke}

Memory managagement is abstracted in Zeke to hide away hardware differences
and obscurities. The page table system in the kernel is relatively light
abstraction but it still allows things that are not usually directly included
in harware implementations, like variable sized page tables.

Each process has its own master page table, in contrast to some kernels where
there might be only one master page table or one partial master page table,
and varying number of level two page tables. The hernel, also known as proc 0,
has its own master page table that is used when a process executes in kernel
mode, as well as when ever a kernel thread is executing. Static or fixed
page table entries are copied to all master page tables created.
A process shares its master page table with its childs on fork() and
exec() will trigger a creation of a new master page table.

\textbf{ARM note:} Only 4 kB pages are used with L2 page tables thus
XN (Execute-Never) bit is always usable also for L2 pages.

\subsection{Domains}

See \verb+MMU_DOM_xxx+ definitions.

\subsection{Virtual memory abstraction levels}

\begin{figure}
\begin{verbatim}
U                      +---------------+
S                      |    malloc     |
R                      +---------------+
------------------------------ | -----------
K   +---------------+  +---------------+
E   |    kmalloc    |  |     proc      |
R   +---------------+  +---------------+
N           |     /\    |      |
E           |     |    \/      |
L  +-----+  |  +---------+   +----+
   | bio |--|--| vralloc |---| vm |
   +-----+  |  +---------+   +----+
            |     |            |
           \/    \/            \/
    +---------------+     +----------+
    |    dynmem     |-----| ptmapper |
    +---------------+     +----------+
            |                  |
           \/                  |
    +---------------+          |
    |    mmu HAL    |<----------
    +---------------+
            |
    +-----------------------+
    | CPU specific MMU code |
    +-----------------------+
----------- | ------------------------------
H   +-------------------+
W   | MMU & coProcessor |
    +-------------------+
\end{verbatim}
\caption{Virtual memory related subsystems in Zeke.}
\label{figure:vmsubsys}
\end{figure}

Virtual memory is managed as virtual memory buffers (\verb+struct buf+) that
are suitable for in-kernel buffers, IO buffers as well as user space memory
mappings. Additionlly the buffer system supports copy-on-write as well as
allocated defined schemes where a part of the memory is stored on a secondary
storage (i.e. paging).

Due to the fact that \verb+buf+ structures are used in different
allocators there is no global knowledge of the actual state of a particular
allocation, instead each allocator should/may keep track of allocation structs
if desired so. Ideally the same struct can be reused when moving data from a
secondary storage allocator to vralloc memory (physical memory). However we
always know wheter a buffer is currently in core or not (\verb+b_data+) and
we also know if a buffer can be swapped to a different allocator
(\verb+B_BUSY+ flag).

See figure \ref{figure:vmsubsys}.

\begin{itemize}
  \item \verb+kmalloc+  - is a kernel level memory allocation service, used
                        solely for memory allocations in kernel space.
  \item \verb+vralloc+  - VRAlloc is a special memory allocator targeted to
                        allocate blocks of memory that will be mapped in virtual
                        address space of a processes it returns \verb+buf+
                        structs instead of just base address of the allocation.
  \item \verb+bio+      - is a IO buffer system, mostly compatible with BSD
                        kernels, utilizing vralloc and buf system.
  \item \verb+dynmem+   - is a dynamic memory allocation system that allocates
                        \& frees contiguous blocks of physical memory (1 MB).
  \item \verb+ptmapper+ - owns all statically allocated page tables
                        (particularly the master page table) and regions,
                        and it is also used to allocate new page tables from
                        the page table region.
  \item \verb+vm+       - vm runs various checks on virtual memory access,
                        copies data between user land, kernel space and
                        allocates and maps memory for processes, and wraps
                        memory mapping operations for proc and \acs{bio}.
  \item mmu HAL -       is an interface to access MMU, provided by \verb+mmu.h+
                        and \verb+mmu.c+.
  \item CPU specific MMU code is the module responsible of configuring the
        physical MMU layer and implementing the HW interface provided by
        \verb+mmu.h+
\end{itemize}


\section{Page Fault handling and VM Region virtualization}

\begin{enumerate}
\item DAB exception transfers execution to \verb+interrupt_dabt+ in \verb+XXX_int_handlers.S+
\item \verb+mmu_data_abort_handler()+ (\verb+XXX_mmu.c+) gets called
\item to be completed...
\end{enumerate}
