\part{Process and Thread Management}

\chapter{Thread Management Concept}

\begin{figure}
\begin{verbatim}
+-------------+
| process_a   |
+-------------+
| pid = pid_a |     +-------------------+
| main_thread |---->| main              |
+-------------+     +-------------------+
                    | id = tid_a        |
                    | pid_owner = pid_a |
                    | inh.parent = NULL |   +---------------------------+
                    | inh.first_child   |-->| child_1                   |
                    | inh.next_child    |   +---------------------------+
                    +--------^----------+   | id = tid_b                |
                             |              | pid_owner = pid_a         |
                             +--------------| inh.parent                |
                             |              | inh.first_child = NULL    |
                             |              | inh.next_child            |---
                             |              +---------------------------+  |
                             |                                             |
                             |              +---------------------------+  |
                             |          --->| child_2                   |<--
                             |          |   +---------------------------+
                             |          |   | id = tid_c                |
                             |          |   | pid_owner = pid_a         |
                             ---------- | --| inh.parent                |
                                        |   | inh.first_child           |---
                                        |   | inh.next_child = NULL     |  |
                                        |   +---------------------------+  |
                                        |                                  |
                                        |   +---------------------------+  |
                                        |   | child_2_1                 |<--
                                        |   +---------------------------+
                                        |   | id = tid_d                |
                                        |   | pid_owner = pid_a         |
                                        ----| inh.parent                |
                                            | inh.first_child = NULL    |
                                            | inh.next_child = NULL     |
                                            +---------------------------+
\end{verbatim}
\caption{Thread tree example.}
\label{figure:thtree}
\end{figure}

\begin{itemize}
  \item \verb+tid_X+ = Thread ID
  \item \verb+pid_a+ = Process ID
\end{itemize}

\verb+process_a+ a has a main thread called \verb+main+. Thread
\verb+main+ has two child thread called \verb+child_1+ and \verb+child_2+.
\verb+child_2+ has created one child thread called \verb+child_2_1+.

\verb+main+ owns all the threads it has created and at the same time child
threads may own their own threads. If parent thread is killed then the
children of the parent are killed first in somewhat reverse order.

\begin{itemize}
  \item \verb+parent+ = Parent thread of the current thread if any
  \item \verb+first_child+ = First child created by the current thread
  \item \verb+next_child+ = Next child in chain of children created by the
        parent thread
\end{itemize}

\section{Cases}
\subparagraph*{process\_a is killed}

Before \verb+process_a+ can be killed \verb+main+ thread must be killed,
because it has child threads its children has to be resolved and killed in
reverse order of creation.

\subparagraph*{child\_2 is killed}

Killing of \verb+child_2+ causes \verb+child_2_1+ to be killed first.


\chapter{Thread Scheduler}

% TODO