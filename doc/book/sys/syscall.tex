\part{System Calls}

\section{Introduction}
User scope stub wrappers are located in \verb+src/usrscope+ directory.
\verb+hal_core.h+ can be included to provide core specific system call
functions.

User scope syscall code is allowed to call the following functions in
\verb+hal_core+ to implement syscall functionality:

\begin{itemize}
  \item \verb+uint32_t syscall(int type, void * p)+
  \item \verb+req_context_switch()+
\end{itemize}

\subsection{Syscall flow}

\begin{enumerate}
\item User scope thread makes a syscall by calling:
      \verb+syscall(SYSCALL_XXX_YYY, &args)+, where XXX is generally a
      module/compilation unit name, YYY is a function name and args is a
      syscall dataset structure in format specified in \verb+syscalldef.h+.

\item After interrupt handler and other hardware specific code execution enters
      to \verb+_intSyscall_handler()+ function where thread local \verb+errno+
      is first reset and then syscall handler of the correct compilation unit is
      resolved according to \verb+syscall_callmap+.

\item Execution enters to compilation unit specific \verb+XXX_syscall()+
      function where it is handled any way wanted, usually by switch\ldots case
      structure and a function call.

\item \verb+XXX_syscall()+ returns a \verb+uint32_t+ value which is, after
      multiple return steps, returned back to the caller which should know
      what type the returned value actually represents.
\end{enumerate}


\subsection{Syscall Major and Minor codes}

System calls are divided to major and minor codes so that major codes represents
a set of functions related to each other, usually all the syscall functions in a
single compilation unit. Major number sets are internally called groups. Both
numbers are defined in \verb+syscall.h+ file.


\section{Adding new syscalls and handlers}

\subsection{A New syscall}

\begin{itemize}
\item \verb+include/syscall.h+ contains syscall number definitions
\item \verb+include/syscalldef.h+ contains some useful structures that can be used when
      creating a new syscall
\item Add the new syscall under a syscall group handler
\end{itemize}

\subsection{A New syscall handler}

\begin{itemize}
\item Create a new syscall group into \verb+include/syscall.h+
\item Create syscall number definitions into the previous file
\item Add the new syscall group to the list of syscall groups in \verb+syscall.c+
\item Create a new syscall group handler
\end{itemize}


\section{sysctl}

The Zeke sysctl mechanism uses hierarchically organized \ac{MIB} tree as a
debugging and online configuration interface to the kernel. This is extremely
useful for example when testing scheduling parameters. Instead of recompiling
after every parameter change it is possible to change kernel's internal
parameters at run time by using sysctl interface.

There is only one syscall for sysctl which handles both reading/writing a
\ac{MIB} variable and queries to the MIB.

\subsection{Magic names}

There is some magic OID's that begins with \verb+{0,...}+ that are used for
queries and other special purposes. Particularly all OID's begin with 0 are
magic names. Currently allocated magic names are described in table
\ref{table:sysctlmagic}.

\begin{table}
\caption{sysctl magic names.}
\label{table:sysctlmagic}
\begin{tabular}{lll}
Name                & Internal function        & Purpose\\
\hline
\verb+{0,1,<iname>}+ & \verb+sysctl_sysctl_name()+     & Get the name of a MIB variable.\\
\verb+{0,2,<iname>}+ & \verb+sysctl_sysctl_next()+     & Get the next variable from MIB tree.\\
\verb+{0,3}+            & \verb+sysctl_sysctl_name2oid()+ & String name to integer name of the variable.\\
\verb+{0,4,<iname>}+ & \verb+sysctl_sysctl_oidfmt()+   & Get format and type of a MIB variable.\\
\verb+{0,5,<iname>}+ & \verb+sysctl_sysctl_oiddescr()+ & Get description string of a MIB variable.
\end{tabular}
\end{table}